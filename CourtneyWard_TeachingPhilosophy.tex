\documentclass[10.5pt]{article}
%\documentstyle[harvard]{article}
\usepackage{setspace, mathrsfs, subfigure}
\usepackage{amsmath, amsthm, amssymb, amsfonts}
\usepackage{rotating, multirow}
\usepackage{fullpage}
\usepackage{ harvard, sectsty}
\bibliographystyle{rfs}
%\usepackage{pdflscape}
\pdfpagewidth 8.5in
%\pdfpageheight 11in 
%\nopagenumbers
%\flushbottom
\topmargin -0.3in
%\topskip 0in
\textheight 9.5in
\textwidth 6.5in
%\oddsidemargin 0.1in
%\evensidemargin 0.1in
%\headheight 0in
%\headsep 0in
%\raggedbottom
%\hypersetup{pdfpagemode=FullScreen}
%
\sectionfont{\large\bf}
\subsectionfont{\normalsize\bf}
\subsubsectionfont{\normalsize\it}
\edef\today{\number\day\space \ifcase\month\or January\or February\or March\or
April\or May\or June\or July\or August\or September\or October\or November\or
December\fi \space \number\year}
\newcommand{\goodgap} {  \hspace{\subfigtopskip} \hspace{\subfigbottomskip} }
%------------------------------------------------------------------------
\begin{document}
%------------------------------------------------------------------------
\singlespace
%\onehalfspace
%\doublespace
%------------------------------------------------------------------------
%------------------------------------------------------------------------

\normalsize{

\hspace{120mm} \textbf{ \normalsize{C}\small{OURTNEY} \normalsize{J.} \normalsize{W}\small{ARD} }
%
% TEACHING BLURB
%
\begin{center}
\textbf{\large{Teaching Philosophy}}\\
\end{center}
{When teaching in any environment, being prepared with organized ideas and a comprehensive knowledge of the subject matter is, in my opinion, the bare minimum. At the end of a course, the average student should be equipped with these qualities. The job of the teacher begins with these qualities and ends with successful communication of ideas to the audience at hand. As a teacher I strive to present more than a textbook account of concepts but to instead develop a clear program of study that I believe is most relevant to the subject matter. Further, when presenting material, I endeavor to provide a context for each topic and gauge the understanding of the student or class of students in real time. To do this well, I keep the following main aspects in mind:}\\
\\ 
\textit{Know your audience: }\\
{Presentations should be tailored to students. In the past, I have taught similar concepts to senior undergraduates and to graduate public policy students but I presented the material in very different ways. For example, the assumptions I can make about knowledge of methods and definitions for the first set of students are broad but I can expect more immediate insights from the latter group. I have found that anticipating these differences can lead to less confusion and more success in student understanding and classroom interaction.}\\
\\
\textit{Motivate topics: }\\
{Students are more engaged when they have been confronted with why a topic is important or relevant. I have also found that it helps to ask questions and garner opinions on a topic before proceeding to exposition of the details. I recently began a lecture by asking students for their best definition of statistics. The answer I received was clear and focused and from a student with no previous background in statistics. In this particular instance it made more of an impression on other students to have a classmate come up with such a succinct description. Further, it provided me a gauge of class understanding and a launching point for the rest of the lecture.}\\
\\
\textit{Provide direction not answers: }\\
{Office hours are always flooded with students who want to know answers. I would rather have office hours attended by students who want guidance in finding answers.  As a teaching assistant for several econometrics course, I noticed that when I provided guidance based on the interests and ideas of the student it resulted in a much better term project than when I suggested one of the stock paper topics and provided programming code as requested.}
\begin{center}
\textbf{\large{Benefits of Teaching}}\\
\end{center}
{My teaching philosophy is closely aligned and supported by my research philosophy. Each aspect of my teaching philosophy also plays a vital role in how I conduct research. Both teaching and research requires effective communication of ideas and moreover, both require that you be able to explain the same concept in several different ways. I have often found this process enlightening in both realms. Perhaps one of the largest benefits I have found from teaching is that through interaction with students, particularly graduate students, I am exposed to topics and ideas that are outside my field. I feel that this has made me a better researcher in the past and I look forward to a future where this is also true.} 
\begin{center}
\textbf{\large{Teaching Interests}}\\
\end{center}
{I have particular teaching interests in microeconomics, econometrics or quantitative methods, labor and health economics. Most recently, my teaching experience is at the graduate level. I designed and delivered a mathematics and statistics course for graduate students in the School of Public Policy at the University of Toronto. In addition, I have acted as a project supervisor and data support resource for econometrics classes in both the Masters of Arts in Economics and the Masters of Financial Economics programs at the University of Toronto. I have also designed and taught a full year senior level undergraduate course in labor economics and have experience as a teaching assistant in microeconomic policy, health economics, public economics, international economics, and macroeconomics. Summary statistics from my most recent course evaluations are attached with the full set of evaluations available upon request.} 

\end{document}
%------------------------------------------------------------------------




