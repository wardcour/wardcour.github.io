\documentclass[10.9pt]{article}
%\documentstyle[harvard]{article}
\usepackage{setspace, mathrsfs, subfigure}
\usepackage{amsmath, amsthm, amssymb, amsfonts}
\usepackage{rotating, multirow}
\usepackage{fullpage}
\usepackage{ harvard, sectsty}
\bibliographystyle{rfs}
%\usepackage{pdflscape}
\pdfpagewidth 8.5in
\pdfpageheight 11in 
%\nopagenumbers
%\flushbottom
\topmargin -0.1in
%\topskip 0in
%\textheight 9.5in
\textwidth 6.4in
%\oddsidemargin 0.1in
%\evensidemargin 0.1in
%\headheight 0in
%\headsep 0in
%\raggedbottom
%\hypersetup{pdfpagemode=FullScreen}
%
\sectionfont{\large\bf}
\subsectionfont{\normalsize\bf}
\subsubsectionfont{\normalsize\it}
\edef\today{\number\day\space \ifcase\month\or January\or February\or March\or
April\or May\or June\or July\or August\or September\or October\or November\or
December\fi \space \number\year}
\newcommand{\goodgap} {  \hspace{\subfigtopskip} \hspace{\subfigbottomskip} }
%------------------------------------------------------------------------
\begin{document}
%------------------------------------------------------------------------
%\singlespace
\onehalfspace
%\doublespace
%------------------------------------------------------------------------
%------------------------------------------------------------------------

\normalsize{

\hspace{120mm} \textbf{ \normalsize{C}\small{OURTNEY} \normalsize{J.} \normalsize{W}\small{ARD} }\\
%
% RESEARCH ABSTRACTS
%
\begin{center}
\textbf{\large{Research Summary}}\\
\end{center}
{My research area is empirical microeconomics with main themes in health and environmental economics.
My dissertation is focused on the theme of public health, most specifically in two dimensions: vaccination and air quality. My job market paper, "�Flu Shots, Work Absences and Hospitalizations: Is an Ounce of Prevention Worth a Pound of Cure?�" investigates a public campaign that delivers free flu shots to healthy children and adults. My subsequent work estimates the relationship between air quality and respiratory health with particular focus on the respiratory health of the very young. Future work will explore the effect of gasoline tax policies on air pollution and policies of smog alerts on avoidance behaviors.
In addition to my work in the area of public health, I am also interested in the effects of insurance schemes on health. In a paper with Mark Stabile; "The Effects of De-listing Publicly Funded Health Care Services," we analyze the demand response across de-listed services, and whether this demand response varies by sub-groups such as low-income and elderly individuals. Along a similar theme, my paper "Public Automobile Insurance, Social Pricing, and Traffic Collisions" explores the impacts of different automobile insurance schemes on demand and further highlights the implications this has for traffic collisions.}\\
\\ 
\textit{Flu Shots, Work Absences, and Hospitalizations: Is an Ounce of Prevention Worth a Pound of Cure?} \\
(Job Market Paper)\\
{In this study, I evaluate the health and economic consequences of a broad-based flu vaccination program. The \textit{Ontario Influenza Immunization Campaign} was introduced in 2001, and delivers free flu shots to healthy children and adults. This program is novel and controversial. Traditionally, the flu shot is recommended only for the elderly or infirm and it is assumed that benefits outside these groups are relatively small. The Ontario flu shot campaign offers a useful policy experiment to evaluate the impact of expansion to children and younger adults. Given that a simple before and after comparison for Ontario may incorrectly attribute all changes in outcomes to the flu shot campaign, and even conventional difference-in-difference comparisons with other provinces may be confounded with differential trends, I instead develop a triple difference identification strategy that exploits variation in the match of the flu shot to the flu. I find that after the campaign, when the vaccine is a good match against circulating strains of flu, Ontario has significantly greater decreases in illness and lost work-time relative to other provinces. Furthermore, based on the results for hospitalizations alone, an ounce of prevention is worth a pound of cure: the program costs approximately \$33 million per year while, in a good match season, the program saves \$144 million in respiratory hospitalization costs. I am also able to provide the first large-scale evidence of the benefits of vaccination in terms of worker productivity. Using labor force data, I find that the program saved an additional \$109 million in worker absence costs. There is also evidence of significant external effects from vaccination. While the results are strongest for children and younger adults, hospitalization rates for those older than 65 fell also even though this group saw no increase in vaccination. This suggests that increased vaccination of the young exerted a positive health externality for the elderly.}\\
\newpage
\hspace{120mm} \textbf{ \normalsize{C}\small{OURTNEY} \normalsize{J.} \normalsize{W}\small{ARD} }\\
\\
\textit{Air Quality, Respiratory Health, and Pollution Policies}\\
{Air pollution is associated with health problems, most notably respiratory problems, in both children and adults. I examine the impact of air pollution on respiratory hospitalizations by using a comprehensive administrative database collecting information on all hospital stays in Canada over an eleven-year period. I combine this with data on several types of pollutants collected from 152 National Air Pollution Surveillance stations across Canada. By doing this, I am able to implement regression models using daily data that control for time and small geographical area fixed effects as well as other observable characteristics such as weather. The relationship between pollution and respiratory health has interesting implications for policies that aim to limit emissions or policies that attempt to limit exposure. For instance, recent fuel taxes (also called transportation taxes), have been put in place in several Canadian cites and may have positive effects for respiratory health if they achieve the intended purpose of reducing vehicular transport. Variation the timing of fuel tax policies allows me to provide evidence on whether this is true. Smog alerts, which have the intended purpose of reducing exposure to ambient pollution, may also have implications for respiratory health. Since alerts are issued at particular threshold levels of pollution, I can exploit a regression discontinuity design to yield conclusions on the effect these alerts have on respiratory health.}\\
\\
\textit{The Effects of De-listing Publicly Funded Health Care Services}
(with Mark Stabile)\\
{This paper uses variation created by the de-listing of insured services across Canadian provinces over the 1990s to estimate both the demand response across services, and whether this demand response varies by sub-groups such as low-income and elderly individuals. Our findings suggest that while the de-listing of services did affect utilization, the effect was not uniform across services, nor across populations. For example, while the demand for physiotherapy and eye exams decreased, the demand for speech therapy services, and chiropractic services increased in some cases. Nor did people adjust along all margins. While the number of people using any physiotherapy services decreased, the number of visits among those who did use physiotherapy services increased. We do not find large difference for low-income individuals with the exception of optometry services, nor do we find large differences between the elderly and the general population with the exception of physiotherapy.}\\
\\
\textit{Public Automobile Insurance and Motor Vehicle Collisions in Canada}\\
{This paper analyses the relationship between motor vehicle collisions and social pricing within a public automobile insurance regime. Social pricing, a type of premium pricing that does not use social factors such as age, gender or marital status in the calculation of insurance rates, is a practice frequently used by public automobile insurers. Because this type of pricing changes the distribution of insured drivers, it has consequences for the overall collision rate. Using the institution of public automobile insurance in two Canadian provinces, this paper finds that after controlling for province and year fixed effects, the collisions rate is 22.2\% higher under public insurance. It also finds that the institution of public insurance changes the number of actuarially riskier drivers insuring vehicles: specifically, younger or elderly drivers. For instance, relative to those aged 35 to 44, drivers that are less than 25 years of age insure 4.6\% more vehicles under social pricing than under private market insurance schemes.} 


\end{document}
%------------------------------------------------------------------------




